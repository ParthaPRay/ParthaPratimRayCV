\documentclass[10pt,a4paper]{article}
\usepackage[utf8]{inputenc}
\usepackage{geometry}
\geometry{margin=0.6in}
\usepackage{hyperref}
\usepackage{enumitem}
\usepackage{titlesec}
\usepackage{array}
\usepackage{newunicodechar}
\newunicodechar{₹}{Rs.}

% Reduce spacing before and after section titles
\titlespacing*{\section}{0pt}{1ex}{0.5ex}
\titlespacing*{\subsection}{0pt}{1ex}{0.5ex}

% Reduce spacing in itemize and enumerate environments
\setlist[itemize]{noitemsep, topsep=0pt, leftmargin=*}
\setlist[enumerate]{noitemsep, topsep=0pt, leftmargin=*}

% Reduce paragraph spacing
\setlength{\parskip}{0pt}
\setlength{\parindent}{0pt}

\begin{document}
	
	\begin{center}
		{\LARGE \textbf{Partha Pratim Ray}}\\
		\vspace{1mm}
		Assistant Professor • Researcher • IoT, Edge Computing and LLM\\
		\vspace{1mm}
		\begin{tabular}{rl}
			\textbf{Email:} & \href{mailto:parthapratimray1986@gmail.com}{parthapratimray1986@gmail.com} \\
			\textbf{GitHub:} & \href{https://github.com/ParthaPRay}{github.com/ParthaPRay} \\
			\textbf{Google Scholar:} & \href{https://scholar.google.co.in/citations?user=ioplfagAAAAJ&hl=en&oi=ao}{scholar.google.co.in/citations?user=ioplfagAAAAJ} \\
			\textbf{Orcid ID:} & \href{http://orcid.org/0000-0003-2306-2792}{0000-0003-2306-2792} \\
			\textbf{Mobile:} & +91-9433668194, +91-7908402850 \\
		\end{tabular}
	\end{center}
	
	\section*{Professional Summary}
	
	Experienced academic with 12+ years of expertise in IoT and Edge Computing, recognized among the world’s top 2\% scientists by Stanford University (2020–2024). Proven skills in developing scalable IoT and edge solutions for smart systems and healthcare. Currently transitioning to specialize in Large Language Models (LLMs) and Machine Learning (ML), focusing on quantized LLMs and open-source AI applications on edge. Seeking to bridge research and practical innovations in AI, ML, and LLMs for real-world challenges. Currently seeking internships and collaborations to deepen expertise in LLM, and ML with the objective of transitioning into the IT industry along with becoming a versatile educator.
	
	
	\section*{Education}
	
	\begin{itemize}[leftmargin=0.15in]
		\item \textbf{Ph.D. in Computer Applications} \hfill \textit{Expected: 2026}\\
		Sikkim University \\
		\textit{Thesis:} Enabling Large Language Models on Resource-Constrained Edge: A Multi-Faceted Approach \\
		\textit{GPA:} 9.29/10
		\item \textbf{M.Tech. in Electronics and Communication Engineering (Embedded Systems)} \hfill \textit{2011}\\
		Haldia Institute of Technology, MAKAUT \\
		\textit{GPA:} 9.60/10
		\item \textbf{B.Tech. in Computer Science and Engineering} \hfill \textit{2008}\\
		B.P. Poddar Institute of Management and Technology, MAKAUT \\
		\textit{GPA:} 8.15/10
		\item \textbf{GATE Qualified} in Computer Science (CS) \hfill \textit{2009}\\
		\textit{Percentile:} 91.62\%
	\end{itemize}
	
	\section*{Key Skills}
	
	\begin{tabular}{ll}
		\textbf{Programming Languages:} &  C, Java, HTML, Python \\
		\textbf{AI/ML/LLM Frameworks:} & \textbf{Familiar:} Ollama, scikit-learn, Pandas, NumPy, Matplotlib, Sentence Transformer\\
		\textbf{Tools and Platforms:} & Raspberry Pi, Arduino, ESP32,  FastAPI, Flask, RESTful API, Quantized LLMs \\
		\textbf{Research \& Analytics:} & IoT, Edge Computing, Large Language Models, Machine Learning Algorithms \\
		\textbf{Certifications:} & Certified Blockchain Expert-V2, Mastering Digital Twins \\
		\textbf{Soft Skills:} & Mentoring, Research \& Analysis, Technical Article Writing \& Publication \\
	\end{tabular}
	
	
	\section*{Key Projects}
	
	
	\textbf{Sequential Agent} \hfill \textit{}\\
	Developed sequential agent using local LLM on resource-constrained edge
	\begin{itemize}[leftmargin=0.2in]
		\item Designed multiple sequential \textit{squad} agents to process user prompts as tasks by allotting some tasks per agent such as 1, 2 , 4 and 8 for agents such as 2, 4, 8, and 16 with collaborative approach.
		\item It used yaml for designing agents and tasks with agents role, goal, backstory, context, output file directory and other configurations. such as all-minilm:33m, redis, Ollama, qwen2.5:0.5b-instruct, FastAPI ad Quad core Cortex-A72.
		
	\end{itemize}
	
	\textbf{Function Call Chaining} \hfill \textit{}\\
	Developed function call chaining using local LLM on resource-constrained edge
	\begin{itemize}[leftmargin=0.2in]
		\item Designed multiple functions to process output of one as input to other function in sequential manner with remote API call and local external functions.
		\item It used Ollama,qwen2.5:0.5b-instruct, llama3.2:1b-instruct-q4\_K\_M, smollm2:1.7b-instruct-q4\_K\_M, all-minilm:33m, FastAPI and Quad core ARM v8 64-bit SoC.
		
	\end{itemize}
	
	\textbf{Sequential Function Call} \hfill \textit{}\\
	Developed function call chaining using local LLM on resource-constrained edge
	\begin{itemize}[leftmargin=0.2in]
		\item Designed multiple functions to process sequential manner with remote API call and local external functions using one-shot and few-shot.
		\item It used Ollama,qwen2.5:0.5b-instruct, llama3.2:1b-instruct-q4\_K\_M, all-minilm:33m, FastAPI ad Quad core  SoC.
		
	\end{itemize}
	
	\textbf{Semantic Static Routing} \hfill \textit{}\\
	Developed static routing using local LLM on resource-constrained edge
	\begin{itemize}[leftmargin=0.2in]
		\item Designed multiple static routes to process user prompt for optimal en-routing to optimal LLM calls.
		\item It used utterance based approach per static route which were compared against sentence transformer such as all-minilm:33m, nomic-embed-text, snowflake-arctic-embed:110m, and mxbai-embed-large
		with Ollama,qwen2.5:0.5b-instruct, FastAPI and Quad core 64-bit SoC.
		
	\end{itemize}
	
	
	\textbf{Semantic Dynamic Routing} \hfill \textit{}\\
	Developed dynamic routing using local LLM on resource-constrained edge
	\begin{itemize}[leftmargin=0.2in]
		\item Designed multiple static routes to process user prompt for optimal en-routing to optimal LLM calls by triggering designated functions using zero-shot, one-shot and few-shot approach with dynamic function schema design.
		\item It used utterance based approach per static route which were compared against sentence transformer such as all-minilm:33m, nomic-embed-text, snowflake-arctic-embed:110m, and mxbai-embed-large
		with Ollama, qwen2:0.5b-instruct, FastAPI and ARM v8. 
		
	\end{itemize}
	
	\textbf{IoT Integration with LLM} \hfill \textit{}\\
	Developed \textit{LLMIoT} and \textit{LLMEdge} - frameworks for integrating IoT with localized LLMs using resource constrained edge scenario.
	\begin{itemize}[leftmargin=0.2in]
		\item Designed framework that employed IoT clients to communicate with local RESTful API based LLMs on constrained edge device.
		\item It used utterance based approach per static route which were compared against sentence transformer such as ESP32, ESP8266 and ARduino MKR1000 with Ollama, qwen2:0.5b, FastAPI and ARM v8 over Wi-Fi. 
		
	\end{itemize}
	
	\textbf{Linguistic Relativity with ChatGPT} \hfill \textit{}\\
	Developed an experiment of hypothesis testing about the applicability of linguistic relativity of various language specific prompts asked to and responded back from ChatGPT 4 mini.
	\begin{itemize}[leftmargin=0.2in]
		\item Designed hypotheses to test whether linguistic relativity is applied on ChatGPT-based multilingual prompt responses.
		\item It used paraphrase-multilingual-MiniLM-L12-v2 sentence transformer to compare semantic similarity across multiple languages responses. Performed \textit{polarity} comparison across the responses of the languages by using \textit{'polyglot'} package. 
		
	\end{itemize}
	
	\textbf{API-aware Image and Video Generation Database System} \hfill \textit{}\\
	Developed a workflow for generating AI-driven, API-aware images and videos based on text prompts, storing the generated content and logs in databases, and serving the results through a web interface.
	\begin{itemize}[leftmargin=0.2in]
		\item Designed a full-stack project to help user getting image (png format) and videos (mp4 format) in local system using user login data.
		\item It used DALL-E-3 for image generation, base64 for converting image download in local machine along with the text prompt to API call RunwayML for 5s video generation and download in local directory. Developed sqlite3, Flask API and sqlalchemy aware databases to manage content and user activity. 
		
	\end{itemize}
	
	\textbf{Retrieval Augmented Generation (RAG)} \hfill \textit{}\\
	Developed a RAG-powered chatbot for legal query resolution based on the Indian Constitution and Indian Penal Code.
	\begin{itemize}[leftmargin=0.2in]
		\item It used five open-source LLMs (Llama3, Mistral, Gemma2, Phi3, and Qwen2) in terms of relevance, faithfulness, context recall, and precision. 
		\item Tools used include LangChain, Ollama, ChromaDB, Streamlit, Python 3.9, and hardware such as an Intel Xeon Processor, 1 TB HDD, and Quadro RTX 5000 GPU. 
		
	\end{itemize}
	
	\section*{Professional Experience}
	
	\textbf{Assistant Professor (Senior Scale)} \hfill \textit{Dec 2020 -- Present}\\
	Department of Computer Applications, Sikkim University, India
	\begin{itemize}[leftmargin=0.2in]
		\item Authored over 100 SCI-indexed journal articles, 32 conference articles, 7 books, 3 book chapters, showcasing a strong commitment to research excellence in areas like IoT, Edge Computing, and LLM.
		\item Supervised numerous master’s thesis projects and contributed to student academic growth with innovative projects such as IoT-based applications, blockchain integration, and edge computing systems.
		\item Played a key role in drafting the syllabus for various programs, including MCA, B.Sc. Computer Science, and vocational courses like B.Voc.
		\item Taught diverse courses ranging from Discrete Mathematics, Data Structures, and Digital Logic to Cybersecurity under NEP 2020 guidelines.
		\item Organized numerous workshops and FDPs, such as: (i) Online FDPs on LaTeX and Moodle LMS (ii) Seminars on Blockchain Technology, Green Computing, and Machine Intelligence.
		Delivered hands-on sessions on IoT, LaTeX, and Raspberry Pi.
		\item Published science communication articles in Dream 2047 on topics like ChatGPT and historical Indian mathematicians.
	\end{itemize}
	
	\textbf{Assistant Professor} \hfill \textit{July 2012 -- Dec 2020}\\
	Department of Computer Applications, Sikkim University, India
	\begin{itemize}[leftmargin=0.2in]
		\item Conducted research in IoT and edge computing, leading to innovative publications and patents filing.
		\item Organized and led workshops and conferences on Blockchain, and IoT technologies.
		\item Contributed to committees like ICT Policy, Admission Working Committee, and NAAC Criterion Search.
		\item Facilitated job and training opportunities through active participation in the Training and Placement Committee.
		\item Served as Organizing Chair for the 6th International Conference on Mathematics and Computing (ICMC) and several TPCs in international-level conferences.
		\item Contributed to holistic student development through initiatives like Innovation and Entrepreneurship Awareness Workshops and Student Clubs and facilitated student participation in technology-driven activities and competitions.
	\end{itemize}
	
	\textbf{Assistant Professor} \hfill \textit{Jan 2012 -- June 2012}\\
	Department of Computer Science and Engineering, Surendra Institute of Engineering and Management, Siliguri, India
	\begin{itemize}[leftmargin=0.2in]
		\item Taught undergraduate courses in C Programming and Embedded Systems.
		\item Developed curriculum and assessment methods for computer science courses.
	\end{itemize}
	
	\section*{Metrics (Dynamic)}
	
	\begin{itemize}[leftmargin=0.2in]
		\item \textbf{Scholar}: H-index 39, i10-index 68, Citations 9883
		\item \textbf{Journals}: SCI Journals 108, SSCI Journals 1, Scopus Journals 21
		\item \textbf{Conference Proceedings}: 36
		\item \textbf{Book Chapters}: 5  
		\item \textbf{Books}: 7
		\item \textbf{Patents Filed}: 7
		\item \textbf{Projects/Dissertations Guided}: 17
	\end{itemize}
	
	\section*{Selected Publications}
	\textit{For a complete list of publications, please visit my \href{https://scholar.google.co.in/citations?user=ioplfagAAAAJ&hl=en&oi=ao}{Google Scholar profile}.}\\
	\textbf{Journal Articles}
	\begin{enumerate}[leftmargin=0.2in]
		\item \textbf{Ray, P. P.}, “ChatGPT: A Comprehensive Review on Background, Applications, Key Challenges, Bias, Ethics, Limitations, and Future Scope,” \textit{Internet of Things and Cyber-Physical Systems}, Elsevier, 2023.
		\item \textbf{Ray, P. P.}, “Benchmarking, Ethical Alignment, and Evaluation Framework for Conversational AI: Advancing Responsible Development of ChatGPT,” \textit{BenchCouncil Transactions on Benchmarks, Standards and Evaluations}, 2023.
		\item \textbf{Ray, P. P.}, “A Review on TinyML: State-of-the-art and Prospects,” \textit{Journal of King Saud University - Computer and Information Sciences}, Elsevier, 2022.
		
	\end{enumerate}
	
	
	\section*{Projects and Consultancy}
	
	\begin{itemize}[leftmargin=0.15in]
		\item \textbf{IoT-Based Plant Temperature Monitoring Systems}, Funded by Sikkim University, 2024--25.
		\item \textbf{Intel IoT Center Setup}, Received Intel Galileo kits from Intel India, 2015--16.
		
	\end{itemize}
	
	\section*{Certifications}
	
	\begin{itemize}[leftmargin=0.15in]
		\item \textbf{Certified Blockchain Expert-V2}, Blockchain Council, 2018.
		\item \textbf{Blockchain and Bitcoin Fundamentals}, Udemy, 2018.
		\item \textbf{Mastering Digital Twins}, Coursera, 2019.
	\end{itemize}
	
	\section*{Awards \& Achievements}
	
	\begin{itemize}[leftmargin=0.15in]
		\item \textbf{Best Paper}, LLMIoT: A Framework for Integration of IoT Devices for Localized Large Language
		Models in the AICTA 2024 at NIT Raipur, 2024.
		\item \textbf{Fellow}, The Institution of Electronics \& Telecommunication Engineers (IETE), 2023.
		\item \textbf{World's Top 2\% Scientist}, Stanford University, 2020--2024.
		\item \textbf{IEI Young Engineers Award}, The Institution of Engineers (India), 2019--20.
		\item \textbf{Emerald Literati Award}, Highly Commended Paper, 2019.
		\item \textbf{Senior Member}, IEEE, since 2019.
		\item \textbf{Young Scientist Award}, Venus International Foundation, 2017.
		\item \textbf{Bharat Vikas Award}, Institute of Self Reliance, 2017.
		\item \textbf{Best Professor in IT Academic Excellence Award}, ICBM-AMP, 2017.
		\item \textbf{Young Achiever Award}, IEAE, 2018.
		\item \textbf{Academia Liaison Officer} of IEEE CTSoc IoT Technical Committee (2024).
		
		
	\end{itemize}
	
	\section*{Professional Memberships}
	
	\begin{itemize}[leftmargin=0.15in]
		\item \textbf{Fellow}, IETE
		\item \textbf{Senior Member}, IEEE
		\item \textbf{Member}, ACM, CSI, IEI, IETE, IET, ISCA, IACSIT
	\end{itemize}
	
	\section*{Additional Information}
	
	\begin{itemize}[leftmargin=0.15in]
		\item \textbf{Languages:} Bengali, Hindi, English
		\item \textbf{Hobbies:} Writing Articles and Books, Open-Source Contribution
		\item \textbf{Founder:} Founded Indian Knowledge Forum (IKF) for Dissemination of Ancient and True Indian Knowledge
	\end{itemize}
	
	\vfill
	
	\begin{center}
		\textit{References and full publication list available upon request.}
	\end{center}
	
\end{document}
